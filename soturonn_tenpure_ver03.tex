\documentclass[uplatex]{jsbook}
\usepackage[dvipdfmx]{graphicx}
\usepackage[dvipdfmx]{color}
\usepackage[dvipdfmx, bookmarkstype=toc, colorlinks=true, pdfborder={0 0 0}, bookmarks=true, bookmarksnumbered=true]{hyperref}%色ついている所はハイパーリンクになっている. 章目次や図目次, 式番号など. 
\usepackage{pxjahyper}
\usepackage{fancyhdr}
\usepackage{here}
\usepackage{wrapfig}
\usepackage{amssymb}
\usepackage{amsmath}
\usepackage{physics}
\usepackage{multicol}
\usepackage{mathtools}
\usepackage{enumerate}
\usepackage{empheq}
\usepackage{url}
\usepackage{multirow}
\usepackage{ulem}
\usepackage{lscape}

%\mathtoolsset{showonlyrefs=true}

\pagestyle{fancy}

 \rfoot{\LaTeX}

 \title{
    \fontsize{25pt}{50pt}\selectfont
    甲南大学\,理工学部\,物理学科\,宇宙理論研究室\\
    \fontsize{25pt}{50pt}\selectfont
    2024年度\,学士論文\\
    \fontsize{25pt}{100pt}\selectfont
    タイトル\,Foge\,Foge
    }
 \author{名前\thanks{Dechapterment of Physics, Faculty of Science and Engineering, Konan University}}
 \date{}%\date{}のみだと日付なし、%\date{}の場合は今日の日付、任意の日付を入れたい時は\date{}の中に任意の日付を書く。


 %-----------------------------------------------
 %|                                             |
 %|   設定は理解しない, 読まない(時間がないので)   |
 %|                                             |
 %-----------------------------------------------


% ここから設定
 
 % 設定1 https://oku.edu.mie-u.ac.jp/~okumura/jsclasses/ "jsbookの余白が広すぎます"から使用
 \setlength{\textwidth}{\fullwidth}
 \setlength{\evensidemargin}{\oddsidemargin}
 % 設定1終わり

 % 設定2 https://oku.edu.mie-u.ac.jp/~okumura/jsclasses/"目次や章の最初のページも他のページと同じデザインにするには"から使用
 \makeatletter
 \renewcommand{\chapter}{%
   \if@openright\cleardoublepage\else\clearpage\fi
   \global\@topnum\z@
   \secdef\@chapter\@schapter}
 \makeatother
 % 設定2終わり

 % 設定3 https://oku.edu.mie-u.ac.jp/~okumura/jsclasses/"目次や章の最初のページも他のページと同じデザインにするには"から使用
 \makeatletter
 \def\ps@plainfoot{%
   \let\@mkboth\@gobbletwo
   \let\@oddhead\@empty
   \def\@oddfoot{\normalfont\hfil-- \thepage\ --\hfil}%
   \let\@evenhead\@empty
   \let\@evenfoot\@oddfoot}
 \let\ps@plain\ps@plainfoot
 \makeatother
 \pagestyle{plain}
 % 設定3終わり

 % 設定4 https://oku.edu.mie-u.ac.jp/~okumura/jsclasses/"目次や章の最初のページも他のページと同じデザインにするには"から使用
 \setlength\footskip{2\baselineskip}
 \addtolength{\textheight}{-2\baselineskip}
 % 設定4終わり

 % 設定5 http://xyoshiki.web.fc2.com/tex/etc010.html
 \makeatletter
 \renewcommand{\theequation}{\arabic{chapter}-\arabic{section}-\arabic{equation}}
 \@addtoreset{equation}{section}
 \makeatother
 % 設定5終わり

 % 設定6
 \renewcommand{\appendixname}{Appendix }%付録をAppendixに表示を変更
 % 設定6終わり

 % 設定7 https://hnsn1202.hateblo.jp/entry/2013/02/06/170617
 %\usepackage{endnotes}
 %\let\footnote=\endnote
 % 設定7終わり

 %注釈を章の最後にまとめて出力する場合は, 以下の3行を表示してほしい所に貼る. 
 %\addcontentsline{toc}{section}{\notesname}
 %\theendnotes
 %\setcounter{endnote}{0}

% 設定はここまで

\begin{document}

 \newcommand{\Eqref}[1]{(\ref{#1})~式}%\Eqref{label名}と入力すると(labelに対応する式番号)式と表示される. 以下2つも同じ. 
 \newcommand{\Tabref}[1]{表~\ref{#1}}
 \newcommand{\Figref}[1]{図~\ref{#1}}
 \newcommand{\Secref}[1]{\ref{#1}~節}

 \maketitle
 \newpage
 
 \begin{abstract}
    ... 文書の概要 ...
 \end{abstract} 
 \newpage

 \setcounter{tocdepth}{3}%目次の設定。章節まで目次に表示の場合->1, 節まで表示の場合->2, 小節まで表示の場合->3
    
 \tableofcontents %目次
 \newpage

 \listoffigures %図目次
 \newpage

 \listoftables %表目次
 \newpage


 \chapter{\LaTeX の書き方}

 ・数式の書き方:式番号を表示させたくない場合はalignをalign*に変更する。

 \begin{align}
  y=ax(コメント:ここに書く)
 \end{align}


 ・分数の書き方:fracの1つ目の括弧は分子、2つ目の括弧は分母

 \begin{align}
  y=\frac{x}{b}
 \end{align}


 ・積分の書き方:dxのみたいなのはdd\{\}中にxだけ書く

 \begin{align}
  \int_{ここから}^{ここまで積分}関数\dd{x}
 \end{align}
 

 ・微分の書き方:\url{https://qiita.com/kawabatayuuya/items/ccfb6f9766f5719b73ca}と\url{https://event.phys.s.u-tokyo.ac.jp/physlab2022/posts/19/}を参考にしてください。\\


 ・連立方程式の書き方:イコールを揃えたい時は"="の前に\& を付ける。また最後の行以外の行の最後にバックスラッシュを付ける。

 \begin{empheq}[left=\empheqlbrace]{align}
    y&=ax\\
    f(x)&=ax+b
 \end{empheq}


 ・特殊文字(ギリシャ文字など):\url{http://www.ic.daito.ac.jp/~mizutani/tex/special_characters.html}\\


 ・添え字の書き方

 \begin{align}
  \Gamma_{下付き添え字}^{上つき添え字または累乗}
 \end{align}


 ・空白の生成:\$ で囲む。

 この$\quad$様にし$\qquad$て書く


 \chapter{序論}\label{chapter 序}



 \chapter{原理}\label{chapter 原理}



 \chapter{解析と結果}\label{chapter 解析と結果}



 \section{議論}\label{section 議論}



 \chapter*{謝辞}
 \addcontentsline{toc}{chapter}{謝辞}%https://blog.ashija.net/2017/10/24/post-2774/
 先生ありがとうございました. みたいな文章を書く. 


 \begin{thebibliography}{99}%https://mathlandscape.com/latex-cite/
   \bibitem{高エネルギー物理学素過程}George B. Rybicki \& Alan P. Lightman, Radiative Processes in Astrophysics, 1985
 \end{thebibliography}


 \appendix
 \chapter{Foge}
 \section{Foge\,Foge}\label{appendix Foge}
 

\end{document}